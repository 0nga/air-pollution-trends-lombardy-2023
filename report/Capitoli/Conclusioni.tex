\section{Conclusions} 

\noindent
After carrying out an initial analysis by dividing the dataset into substances and comparing the different areas, i drew the following conclusions:
\begin{itemize}
	\item the most polluted areas are those in the centre/south of the region, especially in the case of PM10, PM2.5, and nitrogen dioxide (NO2)
	\item for most substances the limits are respected, the most problematic ones are ozone (O3) and PM10. Also lead in the Sondrio province.
\end{itemize}

\noindent
From a second analysis carried out by dividing the dataset into zones, i was able to observe the substances' trends in each individual province and subsequently calculate correlation matrices to understand how the presence of one pollutant can influence others. 
\\By observing the trends obtained for each area, we can say that the substances followed the same trends: 
\begin{itemize}
	\item ozone (O3) tends to increase in the warmer months and decrease in the colder months
	\item all the other substances have an almost opposite trend
\end{itemize}
\noindent
It is possible to confirm this also by observing the tables containing the correlation indices (Tab.~\ref{tab:AQI}): ozone has all negative correlation indices compared to other substances, which implies that as ozone increases, other substances tend to decrease.

\noindent
The limits established by Italian law were respected, with a few exceptions (PM10 and Ozone). However, by consulting other sources of information it's possible to see that the area in northern Italy (the Po Valley) appears to be the most polluted in Italy and also in Europe. 
\\This is mainly due to high vehicular traffic and industry, which emit a large amount of pollutants into the air. The climatic and meteorological conditions of the region also make the situation particularly critical. 
\\In fact, these conditions contribute to increase the concentrations of PM10 and PM2.5 in the air, and make the dispersion of pollutants difficult and slow.

\noindent
Eliminate these emissions is impossible, in the future it is desirable to try to reduce them by tending the concentrations towards zero, in order to preserve the health of the people living in the area.